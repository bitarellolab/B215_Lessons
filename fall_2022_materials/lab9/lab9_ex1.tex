% Options for packages loaded elsewhere
\PassOptionsToPackage{unicode}{hyperref}
\PassOptionsToPackage{hyphens}{url}
%
\documentclass[
]{article}
\usepackage{amsmath,amssymb}
\usepackage{lmodern}
\usepackage{iftex}
\ifPDFTeX
  \usepackage[T1]{fontenc}
  \usepackage[utf8]{inputenc}
  \usepackage{textcomp} % provide euro and other symbols
\else % if luatex or xetex
  \usepackage{unicode-math}
  \defaultfontfeatures{Scale=MatchLowercase}
  \defaultfontfeatures[\rmfamily]{Ligatures=TeX,Scale=1}
\fi
% Use upquote if available, for straight quotes in verbatim environments
\IfFileExists{upquote.sty}{\usepackage{upquote}}{}
\IfFileExists{microtype.sty}{% use microtype if available
  \usepackage[]{microtype}
  \UseMicrotypeSet[protrusion]{basicmath} % disable protrusion for tt fonts
}{}
\makeatletter
\@ifundefined{KOMAClassName}{% if non-KOMA class
  \IfFileExists{parskip.sty}{%
    \usepackage{parskip}
  }{% else
    \setlength{\parindent}{0pt}
    \setlength{\parskip}{6pt plus 2pt minus 1pt}}
}{% if KOMA class
  \KOMAoptions{parskip=half}}
\makeatother
\usepackage{xcolor}
\usepackage[margin=1in]{geometry}
\usepackage{color}
\usepackage{fancyvrb}
\newcommand{\VerbBar}{|}
\newcommand{\VERB}{\Verb[commandchars=\\\{\}]}
\DefineVerbatimEnvironment{Highlighting}{Verbatim}{commandchars=\\\{\}}
% Add ',fontsize=\small' for more characters per line
\usepackage{framed}
\definecolor{shadecolor}{RGB}{248,248,248}
\newenvironment{Shaded}{\begin{snugshade}}{\end{snugshade}}
\newcommand{\AlertTok}[1]{\textcolor[rgb]{0.94,0.16,0.16}{#1}}
\newcommand{\AnnotationTok}[1]{\textcolor[rgb]{0.56,0.35,0.01}{\textbf{\textit{#1}}}}
\newcommand{\AttributeTok}[1]{\textcolor[rgb]{0.77,0.63,0.00}{#1}}
\newcommand{\BaseNTok}[1]{\textcolor[rgb]{0.00,0.00,0.81}{#1}}
\newcommand{\BuiltInTok}[1]{#1}
\newcommand{\CharTok}[1]{\textcolor[rgb]{0.31,0.60,0.02}{#1}}
\newcommand{\CommentTok}[1]{\textcolor[rgb]{0.56,0.35,0.01}{\textit{#1}}}
\newcommand{\CommentVarTok}[1]{\textcolor[rgb]{0.56,0.35,0.01}{\textbf{\textit{#1}}}}
\newcommand{\ConstantTok}[1]{\textcolor[rgb]{0.00,0.00,0.00}{#1}}
\newcommand{\ControlFlowTok}[1]{\textcolor[rgb]{0.13,0.29,0.53}{\textbf{#1}}}
\newcommand{\DataTypeTok}[1]{\textcolor[rgb]{0.13,0.29,0.53}{#1}}
\newcommand{\DecValTok}[1]{\textcolor[rgb]{0.00,0.00,0.81}{#1}}
\newcommand{\DocumentationTok}[1]{\textcolor[rgb]{0.56,0.35,0.01}{\textbf{\textit{#1}}}}
\newcommand{\ErrorTok}[1]{\textcolor[rgb]{0.64,0.00,0.00}{\textbf{#1}}}
\newcommand{\ExtensionTok}[1]{#1}
\newcommand{\FloatTok}[1]{\textcolor[rgb]{0.00,0.00,0.81}{#1}}
\newcommand{\FunctionTok}[1]{\textcolor[rgb]{0.00,0.00,0.00}{#1}}
\newcommand{\ImportTok}[1]{#1}
\newcommand{\InformationTok}[1]{\textcolor[rgb]{0.56,0.35,0.01}{\textbf{\textit{#1}}}}
\newcommand{\KeywordTok}[1]{\textcolor[rgb]{0.13,0.29,0.53}{\textbf{#1}}}
\newcommand{\NormalTok}[1]{#1}
\newcommand{\OperatorTok}[1]{\textcolor[rgb]{0.81,0.36,0.00}{\textbf{#1}}}
\newcommand{\OtherTok}[1]{\textcolor[rgb]{0.56,0.35,0.01}{#1}}
\newcommand{\PreprocessorTok}[1]{\textcolor[rgb]{0.56,0.35,0.01}{\textit{#1}}}
\newcommand{\RegionMarkerTok}[1]{#1}
\newcommand{\SpecialCharTok}[1]{\textcolor[rgb]{0.00,0.00,0.00}{#1}}
\newcommand{\SpecialStringTok}[1]{\textcolor[rgb]{0.31,0.60,0.02}{#1}}
\newcommand{\StringTok}[1]{\textcolor[rgb]{0.31,0.60,0.02}{#1}}
\newcommand{\VariableTok}[1]{\textcolor[rgb]{0.00,0.00,0.00}{#1}}
\newcommand{\VerbatimStringTok}[1]{\textcolor[rgb]{0.31,0.60,0.02}{#1}}
\newcommand{\WarningTok}[1]{\textcolor[rgb]{0.56,0.35,0.01}{\textbf{\textit{#1}}}}
\usepackage{graphicx}
\makeatletter
\def\maxwidth{\ifdim\Gin@nat@width>\linewidth\linewidth\else\Gin@nat@width\fi}
\def\maxheight{\ifdim\Gin@nat@height>\textheight\textheight\else\Gin@nat@height\fi}
\makeatother
% Scale images if necessary, so that they will not overflow the page
% margins by default, and it is still possible to overwrite the defaults
% using explicit options in \includegraphics[width, height, ...]{}
\setkeys{Gin}{width=\maxwidth,height=\maxheight,keepaspectratio}
% Set default figure placement to htbp
\makeatletter
\def\fps@figure{htbp}
\makeatother
\setlength{\emergencystretch}{3em} % prevent overfull lines
\providecommand{\tightlist}{%
  \setlength{\itemsep}{0pt}\setlength{\parskip}{0pt}}
\setcounter{secnumdepth}{-\maxdimen} % remove section numbering
\ifLuaTeX
  \usepackage{selnolig}  % disable illegal ligatures
\fi
\IfFileExists{bookmark.sty}{\usepackage{bookmark}}{\usepackage{hyperref}}
\IfFileExists{xurl.sty}{\usepackage{xurl}}{} % add URL line breaks if available
\urlstyle{same} % disable monospaced font for URLs
\hypersetup{
  pdftitle={Hospital readmission rates of acute ischemic stroke in California},
  hidelinks,
  pdfcreator={LaTeX via pandoc}}

\title{Hospital readmission rates of acute ischemic stroke in
California}
\author{}
\date{\vspace{-2.5em}}

\begin{document}
\maketitle

Tip: you can click on \texttt{visual} above this document to experience
a friendlier interface.

\hypertarget{motivation}{%
\subsection{Motivation}\label{motivation}}

According to the Nationwide Readmissions Database of the Healthcare Cost
and Utilization Project between 2010 and 2015, the 30-day hospital
readmission rate for acute ischemic stroke patients on a national level
is 12.4\% (Bambhroliya et al.~2018). A researcher wants to test whether
the proportion of 30-day hospital readmissions for a California hospital
with an ``as expected'' hospital quality rating differs from the
national 30-day readmission proportion.

\hypertarget{data}{%
\subsection{Data}\label{data}}

CSV Data file: \emph{readmin.csv}

The data file contains ischemic stroke 30-day hospital readmission
incidence data for a random sample of patients in a California hospital
with an ``as expected'' quality rating obtained from a set of hospital
records for 2014-2015. The 30-day readmission data from a sample of 50
patients was recorded.

Read in the dataset, which we'll call \emph{readmin}. Use \texttt{head}
to check what it looks like:

\begin{Shaded}
\begin{Highlighting}[]
\CommentTok{\# read dataset}
\CommentTok{\# use head()}
\end{Highlighting}
\end{Shaded}

The variable \emph{ReadmissionStatus} is a binary variable which is
equal to 1 if the patient was readmitted to the hospital within 30 days
of discharge and equal to 0 if the patient was not readmitted to the
hospital within 30 days of discharge.

\hypertarget{questions-of-interest}{%
\subsection{Questions of interest}\label{questions-of-interest}}

Two questions of interest are:

\begin{enumerate}
\def\labelenumi{\arabic{enumi}.}
\item
  Is the proportion of stroke patients readmitted within 30-days of
  discharge from CA hospital different from the nationwide proportion?
\item
  What is the 95\% confidence interval for the proportion of acute
  ischemic stroke patients readmitted within 30 days of discharge?
\end{enumerate}

\hypertarget{instructions}{%
\subsection{Instructions}\label{instructions}}

\begin{itemize}
\tightlist
\item
  Install the \texttt{binom()} package and load it:
\end{itemize}

\begin{Shaded}
\begin{Highlighting}[]
\CommentTok{\# Install}
\CommentTok{\# load}
\end{Highlighting}
\end{Shaded}

\hypertarget{exploring-the-data}{%
\subsection{Exploring the Data}\label{exploring-the-data}}

Let's begin by looking at the variable of interest, readmission. Use the
\texttt{table()} function.

\begin{Shaded}
\begin{Highlighting}[]
\CommentTok{\# Table of Readmission Status}
\end{Highlighting}
\end{Shaded}

\emph{How many peope are readmitted to the hospital within 30 days? How
many are not?}

\hypertarget{data-analysis}{%
\subsection{Data Analysis}\label{data-analysis}}

Determine the proportion of individuals readmitted in the sample.
\emph{Tip: check out the \texttt{prop.table()} function.}

\begin{Shaded}
\begin{Highlighting}[]
\CommentTok{\# Proportion table of Readmission Status}
\end{Highlighting}
\end{Shaded}

Answer: of the sample was readmitted within 30 days.

\hypertarget{the-binomial-test}{%
\paragraph{The Binomial test}\label{the-binomial-test}}

Write the null and alternative hypotheses for this statistical test,
comparing the proportion of readmission in our study, to the population
proportion of 0.124.

\(H_0: \text{The proportion of readmission in the population is }______\)

\(H_A: \text{The proportion of readmission in the population is not }_______\)

Run a two-sided binomial test to see if the sample proportion differs
significantly from the population proportion. \emph{Tip: Some
possibilities were covered in lectures. You can also just check out the
\texttt{binom} package now.}

\begin{Shaded}
\begin{Highlighting}[]
\CommentTok{\# Binomial test of readmission status}
\end{Highlighting}
\end{Shaded}

\emph{What is your decision based on the binomial test?}

Answer:

\hypertarget{agresti-coull-95-confidence-interval-for-the-proportion}{%
\paragraph{Agresti-Coull 95\% Confidence Interval for the
proportion}\label{agresti-coull-95-confidence-interval-for-the-proportion}}

Next, calculate the 95\% CI for the proportion of stroke patients who
were readmitted within 30 days of discharge using the Agresti-Coull
method. \emph{Tip: Some possibilities were covered in lectures. You can
also just check out the \texttt{binom} package now.}

\begin{Shaded}
\begin{Highlighting}[]
\CommentTok{\# 95\% CI for binomial test}
\end{Highlighting}
\end{Shaded}

\emph{What is the Agresti-Coull 95\% CI?}

Answer:

\hypertarget{conclusions}{%
\subsection{Conclusions}\label{conclusions}}

Based on these results, we find no evidence that the population
proportion is significantly different from 0.124. The 95\% CI interval
(\(0.0525 < p < 0.2417\)) indicates medium level of precision, given
that it covers nearly 20\% of possible proportions.

\hypertarget{references}{%
\subsection{References}\label{references}}

Bambhroliya AB, Donnelly JP, Thomas EJ, et al., ``Estimates and Temporal
Trend for US Nationwide 30-Day Hospital Readmission Among Patients With
Ischemic and Hemorrhagic Stroke.'', \emph{JAMA Netw Open},
1{[}2018{]}:e181190
\url{https://jamanetwork.com/journals/jamanetworkopen/fullarticle/2696869}

\end{document}
