% Options for packages loaded elsewhere
\PassOptionsToPackage{unicode}{hyperref}
\PassOptionsToPackage{hyphens}{url}
%
\documentclass[
]{article}
\usepackage{amsmath,amssymb}
\usepackage{lmodern}
\usepackage{iftex}
\ifPDFTeX
  \usepackage[T1]{fontenc}
  \usepackage[utf8]{inputenc}
  \usepackage{textcomp} % provide euro and other symbols
\else % if luatex or xetex
  \usepackage{unicode-math}
  \defaultfontfeatures{Scale=MatchLowercase}
  \defaultfontfeatures[\rmfamily]{Ligatures=TeX,Scale=1}
\fi
% Use upquote if available, for straight quotes in verbatim environments
\IfFileExists{upquote.sty}{\usepackage{upquote}}{}
\IfFileExists{microtype.sty}{% use microtype if available
  \usepackage[]{microtype}
  \UseMicrotypeSet[protrusion]{basicmath} % disable protrusion for tt fonts
}{}
\makeatletter
\@ifundefined{KOMAClassName}{% if non-KOMA class
  \IfFileExists{parskip.sty}{%
    \usepackage{parskip}
  }{% else
    \setlength{\parindent}{0pt}
    \setlength{\parskip}{6pt plus 2pt minus 1pt}}
}{% if KOMA class
  \KOMAoptions{parskip=half}}
\makeatother
\usepackage{xcolor}
\usepackage[margin=1in]{geometry}
\usepackage{color}
\usepackage{fancyvrb}
\newcommand{\VerbBar}{|}
\newcommand{\VERB}{\Verb[commandchars=\\\{\}]}
\DefineVerbatimEnvironment{Highlighting}{Verbatim}{commandchars=\\\{\}}
% Add ',fontsize=\small' for more characters per line
\usepackage{framed}
\definecolor{shadecolor}{RGB}{248,248,248}
\newenvironment{Shaded}{\begin{snugshade}}{\end{snugshade}}
\newcommand{\AlertTok}[1]{\textcolor[rgb]{0.94,0.16,0.16}{#1}}
\newcommand{\AnnotationTok}[1]{\textcolor[rgb]{0.56,0.35,0.01}{\textbf{\textit{#1}}}}
\newcommand{\AttributeTok}[1]{\textcolor[rgb]{0.77,0.63,0.00}{#1}}
\newcommand{\BaseNTok}[1]{\textcolor[rgb]{0.00,0.00,0.81}{#1}}
\newcommand{\BuiltInTok}[1]{#1}
\newcommand{\CharTok}[1]{\textcolor[rgb]{0.31,0.60,0.02}{#1}}
\newcommand{\CommentTok}[1]{\textcolor[rgb]{0.56,0.35,0.01}{\textit{#1}}}
\newcommand{\CommentVarTok}[1]{\textcolor[rgb]{0.56,0.35,0.01}{\textbf{\textit{#1}}}}
\newcommand{\ConstantTok}[1]{\textcolor[rgb]{0.00,0.00,0.00}{#1}}
\newcommand{\ControlFlowTok}[1]{\textcolor[rgb]{0.13,0.29,0.53}{\textbf{#1}}}
\newcommand{\DataTypeTok}[1]{\textcolor[rgb]{0.13,0.29,0.53}{#1}}
\newcommand{\DecValTok}[1]{\textcolor[rgb]{0.00,0.00,0.81}{#1}}
\newcommand{\DocumentationTok}[1]{\textcolor[rgb]{0.56,0.35,0.01}{\textbf{\textit{#1}}}}
\newcommand{\ErrorTok}[1]{\textcolor[rgb]{0.64,0.00,0.00}{\textbf{#1}}}
\newcommand{\ExtensionTok}[1]{#1}
\newcommand{\FloatTok}[1]{\textcolor[rgb]{0.00,0.00,0.81}{#1}}
\newcommand{\FunctionTok}[1]{\textcolor[rgb]{0.00,0.00,0.00}{#1}}
\newcommand{\ImportTok}[1]{#1}
\newcommand{\InformationTok}[1]{\textcolor[rgb]{0.56,0.35,0.01}{\textbf{\textit{#1}}}}
\newcommand{\KeywordTok}[1]{\textcolor[rgb]{0.13,0.29,0.53}{\textbf{#1}}}
\newcommand{\NormalTok}[1]{#1}
\newcommand{\OperatorTok}[1]{\textcolor[rgb]{0.81,0.36,0.00}{\textbf{#1}}}
\newcommand{\OtherTok}[1]{\textcolor[rgb]{0.56,0.35,0.01}{#1}}
\newcommand{\PreprocessorTok}[1]{\textcolor[rgb]{0.56,0.35,0.01}{\textit{#1}}}
\newcommand{\RegionMarkerTok}[1]{#1}
\newcommand{\SpecialCharTok}[1]{\textcolor[rgb]{0.00,0.00,0.00}{#1}}
\newcommand{\SpecialStringTok}[1]{\textcolor[rgb]{0.31,0.60,0.02}{#1}}
\newcommand{\StringTok}[1]{\textcolor[rgb]{0.31,0.60,0.02}{#1}}
\newcommand{\VariableTok}[1]{\textcolor[rgb]{0.00,0.00,0.00}{#1}}
\newcommand{\VerbatimStringTok}[1]{\textcolor[rgb]{0.31,0.60,0.02}{#1}}
\newcommand{\WarningTok}[1]{\textcolor[rgb]{0.56,0.35,0.01}{\textbf{\textit{#1}}}}
\usepackage{graphicx}
\makeatletter
\def\maxwidth{\ifdim\Gin@nat@width>\linewidth\linewidth\else\Gin@nat@width\fi}
\def\maxheight{\ifdim\Gin@nat@height>\textheight\textheight\else\Gin@nat@height\fi}
\makeatother
% Scale images if necessary, so that they will not overflow the page
% margins by default, and it is still possible to overwrite the defaults
% using explicit options in \includegraphics[width, height, ...]{}
\setkeys{Gin}{width=\maxwidth,height=\maxheight,keepaspectratio}
% Set default figure placement to htbp
\makeatletter
\def\fps@figure{htbp}
\makeatother
\setlength{\emergencystretch}{3em} % prevent overfull lines
\providecommand{\tightlist}{%
  \setlength{\itemsep}{0pt}\setlength{\parskip}{0pt}}
\setcounter{secnumdepth}{-\maxdimen} % remove section numbering
\ifLuaTeX
  \usepackage{selnolig}  % disable illegal ligatures
\fi
\IfFileExists{bookmark.sty}{\usepackage{bookmark}}{\usepackage{hyperref}}
\IfFileExists{xurl.sty}{\usepackage{xurl}}{} % add URL line breaks if available
\urlstyle{same} % disable monospaced font for URLs
\hypersetup{
  pdftitle={Cheatsheet},
  hidelinks,
  pdfcreator={LaTeX via pandoc}}

\title{Cheatsheet}
\author{}
\date{\vspace{-2.5em}2022-11-02}

\begin{document}
\maketitle

\hypertarget{what-is-r}{%
\subsection{What is R?}\label{what-is-r}}

\begin{itemize}
\tightlist
\item
  R is a free, open-source statistical programming language
\item
  Many programming languages exist: E.g. java script, C, C++, Python,
  etc.
\item
  R is widely used by scientists worldwide
\item
  You can use R to do everything: calculating simple summary statistics,
  performing complex simulations, creating gorgeous plots
\end{itemize}

\hypertarget{what-is-rstudio}{%
\subsection{What is Rstudio}\label{what-is-rstudio}}

\includegraphics{images/Screen Shot 2022-10-19 at 12.45.58 PM.png}

Rstudio is an IDE - integrated development environment that allows you
to write and run R code, visualize figures produced in R, and many other
neat things

\begin{itemize}
\tightlist
\item
  Allows one to combine R code, analyses, plots, and written text into
  elegant documents all in one place using \emph{Rmarkdown}
\end{itemize}

\hypertarget{r-as-a-calculator}{%
\subsection{R as a calculator}\label{r-as-a-calculator}}

\begin{Shaded}
\begin{Highlighting}[]
\DecValTok{2}\SpecialCharTok{+}\DecValTok{2} \CommentTok{\#addition}
\end{Highlighting}
\end{Shaded}

\begin{verbatim}
## [1] 4
\end{verbatim}

\begin{Shaded}
\begin{Highlighting}[]
\DecValTok{534{-}430} \CommentTok{\#subtraction}
\end{Highlighting}
\end{Shaded}

\begin{verbatim}
## [1] 104
\end{verbatim}

\begin{Shaded}
\begin{Highlighting}[]
\DecValTok{128421847}\SpecialCharTok{*}\DecValTok{3} \CommentTok{\#multiplication}
\end{Highlighting}
\end{Shaded}

\begin{verbatim}
## [1] 385265541
\end{verbatim}

\begin{Shaded}
\begin{Highlighting}[]
\DecValTok{12819482}\SpecialCharTok{/}\DecValTok{17} \CommentTok{\#division}
\end{Highlighting}
\end{Shaded}

\begin{verbatim}
## [1] 754087.2
\end{verbatim}

\hypertarget{r-as-a-calculator-1}{%
\subsection{R as a calculator}\label{r-as-a-calculator-1}}

\begin{Shaded}
\begin{Highlighting}[]
\DecValTok{4} \SpecialCharTok{\^{}} \DecValTok{3} \CommentTok{\#exponent}
\end{Highlighting}
\end{Shaded}

\begin{verbatim}
## [1] 64
\end{verbatim}

\begin{Shaded}
\begin{Highlighting}[]
\DecValTok{4} \SpecialCharTok{**} \DecValTok{3} \CommentTok{\#exponent (also works)}
\end{Highlighting}
\end{Shaded}

\begin{verbatim}
## [1] 64
\end{verbatim}

\begin{Shaded}
\begin{Highlighting}[]
\DecValTok{4} \SpecialCharTok{\%\%} \DecValTok{3} \CommentTok{\#modulo: the remainder of a division}
\end{Highlighting}
\end{Shaded}

\begin{verbatim}
## [1] 1
\end{verbatim}

\hypertarget{r-for-logical-operations}{%
\subsection{R for logical operations}\label{r-for-logical-operations}}

\begin{Shaded}
\begin{Highlighting}[]
\DecValTok{4} \SpecialCharTok{==} \DecValTok{3} \CommentTok{\#equality}
\end{Highlighting}
\end{Shaded}

\begin{verbatim}
## [1] FALSE
\end{verbatim}

\begin{Shaded}
\begin{Highlighting}[]
\DecValTok{4} \SpecialCharTok{!=} \DecValTok{3} \CommentTok{\#non{-}equality}
\end{Highlighting}
\end{Shaded}

\begin{verbatim}
## [1] TRUE
\end{verbatim}

\begin{Shaded}
\begin{Highlighting}[]
\DecValTok{4} \SpecialCharTok{\textless{}} \DecValTok{3} \CommentTok{\#logical, lower than}
\end{Highlighting}
\end{Shaded}

\begin{verbatim}
## [1] FALSE
\end{verbatim}

\begin{Shaded}
\begin{Highlighting}[]
\DecValTok{4} \SpecialCharTok{\textgreater{}=} \DecValTok{3} \CommentTok{\#logical, greater than or equal to}
\end{Highlighting}
\end{Shaded}

\begin{verbatim}
## [1] TRUE
\end{verbatim}

\begin{Shaded}
\begin{Highlighting}[]
\ConstantTok{TRUE} \SpecialCharTok{==} \DecValTok{1} \CommentTok{\#this may surprise you!}
\end{Highlighting}
\end{Shaded}

\begin{verbatim}
## [1] TRUE
\end{verbatim}

\hypertarget{assigning-values-to-r-objects}{%
\subsection{Assigning values to R
objects}\label{assigning-values-to-r-objects}}

\begin{Shaded}
\begin{Highlighting}[]
\NormalTok{x }\OtherTok{\textless{}{-}} \DecValTok{4} \CommentTok{\# assign 4 to R object x}

\NormalTok{x }\CommentTok{\#print object x}
\end{Highlighting}
\end{Shaded}

\begin{verbatim}
## [1] 4
\end{verbatim}

\begin{Shaded}
\begin{Highlighting}[]
\NormalTok{y }\OtherTok{=} \DecValTok{4} \CommentTok{\# assign 4 to R object y}

\NormalTok{y }\CommentTok{\#print object y}
\end{Highlighting}
\end{Shaded}

\begin{verbatim}
## [1] 4
\end{verbatim}

\begin{Shaded}
\begin{Highlighting}[]
\NormalTok{x }\SpecialCharTok{==}\NormalTok{ y }\CommentTok{\#logical, equality}
\end{Highlighting}
\end{Shaded}

\begin{verbatim}
## [1] TRUE
\end{verbatim}

\hypertarget{data-types-in-r}{%
\subsection{Data types in R}\label{data-types-in-r}}

\begin{Shaded}
\begin{Highlighting}[]
\NormalTok{x }\OtherTok{\textless{}{-}} \DecValTok{12} \CommentTok{\# assign number 12 to x}

\FunctionTok{class}\NormalTok{(x) }\CommentTok{\#check class or data type}
\end{Highlighting}
\end{Shaded}

\begin{verbatim}
## [1] "numeric"
\end{verbatim}

\begin{Shaded}
\begin{Highlighting}[]
\NormalTok{x1 }\OtherTok{\textless{}{-}} \StringTok{"12"} \CommentTok{\# assign string 12 to x}

\FunctionTok{class}\NormalTok{(x1) }\CommentTok{\#check class or data type}
\end{Highlighting}
\end{Shaded}

\begin{verbatim}
## [1] "character"
\end{verbatim}

\begin{Shaded}
\begin{Highlighting}[]
\NormalTok{y }\OtherTok{\textless{}{-}} \ConstantTok{FALSE} \CommentTok{\# assign boolean FALSE to y}

\FunctionTok{class}\NormalTok{(y) }\CommentTok{\#check class or data type}
\end{Highlighting}
\end{Shaded}

\begin{verbatim}
## [1] "logical"
\end{verbatim}

\begin{Shaded}
\begin{Highlighting}[]
\NormalTok{y1 }\OtherTok{\textless{}{-}} \StringTok{"FALSE"} \CommentTok{\# assign string "FALSE" to y}

\FunctionTok{class}\NormalTok{(y1) }\CommentTok{\#check class or data type}
\end{Highlighting}
\end{Shaded}

\begin{verbatim}
## [1] "character"
\end{verbatim}

\hypertarget{data-structures}{%
\subsection{Data Structures}\label{data-structures}}

\hypertarget{vectors}{%
\subsubsection{Vectors}\label{vectors}}

All elements must be of the same type.

\begin{Shaded}
\begin{Highlighting}[]
\NormalTok{x }\OtherTok{\textless{}{-}} \FunctionTok{c}\NormalTok{(}\DecValTok{12}\NormalTok{,}\DecValTok{13}\NormalTok{,}\DecValTok{1}\NormalTok{,}\DecValTok{5765}\NormalTok{,}\DecValTok{12}\NormalTok{) }\CommentTok{\# concatenate numbers and assign to x}

\FunctionTok{is.vector}\NormalTok{(x) }\CommentTok{\#logical}
\end{Highlighting}
\end{Shaded}

\begin{verbatim}
## [1] TRUE
\end{verbatim}

\begin{Shaded}
\begin{Highlighting}[]
\FunctionTok{class}\NormalTok{(x) }\CommentTok{\#check class or data type}
\end{Highlighting}
\end{Shaded}

\begin{verbatim}
## [1] "numeric"
\end{verbatim}

\begin{Shaded}
\begin{Highlighting}[]
\FunctionTok{print}\NormalTok{(x)}
\end{Highlighting}
\end{Shaded}

\begin{verbatim}
## [1]   12   13    1 5765   12
\end{verbatim}

\hypertarget{data-structures-1}{%
\subsection{Data Structures}\label{data-structures-1}}

\hypertarget{vectors-1}{%
\subsubsection{Vectors}\label{vectors-1}}

All elements must be of the same type.

\begin{Shaded}
\begin{Highlighting}[]
\NormalTok{x1 }\OtherTok{\textless{}{-}} \FunctionTok{c}\NormalTok{(}\StringTok{"Pumpkin"}\NormalTok{,}\DecValTok{13}\NormalTok{,}\DecValTok{1}\NormalTok{,}\DecValTok{5765}\NormalTok{,}\DecValTok{12}\NormalTok{) }\CommentTok{\# concatenate numbers and assign to x}

\FunctionTok{is.vector}\NormalTok{(x1) }\CommentTok{\#logical}
\end{Highlighting}
\end{Shaded}

\begin{verbatim}
## [1] TRUE
\end{verbatim}

\begin{Shaded}
\begin{Highlighting}[]
\FunctionTok{class}\NormalTok{(x1) }\CommentTok{\#check class or data type}
\end{Highlighting}
\end{Shaded}

\begin{verbatim}
## [1] "character"
\end{verbatim}

\begin{Shaded}
\begin{Highlighting}[]
\FunctionTok{print}\NormalTok{(x1)}
\end{Highlighting}
\end{Shaded}

\begin{verbatim}
## [1] "Pumpkin" "13"      "1"       "5765"    "12"
\end{verbatim}

\hypertarget{data-structures-2}{%
\subsection{Data Structures}\label{data-structures-2}}

\hypertarget{lists}{%
\subsubsection{Lists}\label{lists}}

Elements can be of different types

\begin{Shaded}
\begin{Highlighting}[]
\NormalTok{x1 }\OtherTok{\textless{}{-}} \FunctionTok{c}\NormalTok{(}\StringTok{"Pumpkin"}\NormalTok{,}\DecValTok{13}\NormalTok{,}\DecValTok{1}\NormalTok{,}\DecValTok{5765}\NormalTok{,}\DecValTok{12}\NormalTok{) }\CommentTok{\# concatenate numbers and assign to x}

\FunctionTok{is.vector}\NormalTok{(x1) }\CommentTok{\#logical}
\end{Highlighting}
\end{Shaded}

\begin{verbatim}
## [1] TRUE
\end{verbatim}

\begin{Shaded}
\begin{Highlighting}[]
\FunctionTok{class}\NormalTok{(x1) }\CommentTok{\#check class or data type}
\end{Highlighting}
\end{Shaded}

\begin{verbatim}
## [1] "character"
\end{verbatim}

\begin{Shaded}
\begin{Highlighting}[]
\FunctionTok{print}\NormalTok{(x1)}
\end{Highlighting}
\end{Shaded}

\begin{verbatim}
## [1] "Pumpkin" "13"      "1"       "5765"    "12"
\end{verbatim}

\hypertarget{data-structures-3}{%
\subsection{Data Structures}\label{data-structures-3}}

\hypertarget{matrix}{%
\subsubsection{Matrix}\label{matrix}}

All elements must be of the same type. Two dimensiosn

\begin{Shaded}
\begin{Highlighting}[]
\NormalTok{mat}\OtherTok{\textless{}{-}}\FunctionTok{matrix}\NormalTok{(}\FunctionTok{rnorm}\NormalTok{(}\AttributeTok{n=}\DecValTok{100}\NormalTok{, }\AttributeTok{mean=}\DecValTok{0}\NormalTok{, }\AttributeTok{sd=}\DecValTok{1}\NormalTok{),}\AttributeTok{nrow=}\DecValTok{2}\NormalTok{) }\CommentTok{\#sample 100 numbers from normal distribution}
\CommentTok{\#with mean 0 and sd 1}
\CommentTok{\#use them to make a matrix of two rows (50 cols by extension)}
\FunctionTok{dim}\NormalTok{(mat)}
\end{Highlighting}
\end{Shaded}

\begin{verbatim}
## [1]  2 50
\end{verbatim}

\hypertarget{general-functions}{%
\subsubsection{General functions:}\label{general-functions}}

\texttt{group\_by()}: Conduct operations separately by values of a
column (or columns).

\texttt{summarise()}: Reduces our data from the many observations for
each variable to just the summaries we ask for. Summaries will be one
row long if we have not group\_by() anything, or the number of groups if
we have.

\texttt{sum()}: Adding up all values in a vector.

\texttt{diff()}: Subtract sequential entries in a vector.

\texttt{sqrt()}: Find the square root of all entries in a vector.

\texttt{unique()}: Reduce a vector to only its unique values.

\texttt{pull()}: Extract a column from a tibble as a vector.

\texttt{round()}: Round values in a vector to the specified number of
digits.

\hypertarget{summarizing-location-center}{%
\subsubsection{Summarizing location
(center):}\label{summarizing-location-center}}

\texttt{n()}: The size of a sample.

\texttt{mean()}: The mean of a variable in our sample.

\texttt{median()}: The median of a variable in our sample.

\hypertarget{summarizing-width-spread}{%
\subsubsection{Summarizing width
(spread):}\label{summarizing-width-spread}}

\texttt{max()}: The largest value in a vector.

\texttt{min()}: The smallest value in a vector.

\texttt{range()}: Find the smallest and larges values in a vector.
Combine with diff() to find the difference between the largest and
smallest value.

\texttt{quantile()}: Find values in a give quantile. Use
\texttt{diff(quantile(...\ ,\ probs\ =\ c(0.25,\ 0.75)))} to find the
interquartile range.

\texttt{IQR()}: Find the difference between the third and first quartile
(aka the interquartile range).

\texttt{var()}: Find the sample variance of vector.

\texttt{sd()}: Find the sample standard deviation of vector.

\end{document}
