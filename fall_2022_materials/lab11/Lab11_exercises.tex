% Options for packages loaded elsewhere
\PassOptionsToPackage{unicode}{hyperref}
\PassOptionsToPackage{hyphens}{url}
%
\documentclass[
]{article}
\usepackage{amsmath,amssymb}
\usepackage{iftex}
\ifPDFTeX
  \usepackage[T1]{fontenc}
  \usepackage[utf8]{inputenc}
  \usepackage{textcomp} % provide euro and other symbols
\else % if luatex or xetex
  \usepackage{unicode-math} % this also loads fontspec
  \defaultfontfeatures{Scale=MatchLowercase}
  \defaultfontfeatures[\rmfamily]{Ligatures=TeX,Scale=1}
\fi
\usepackage{lmodern}
\ifPDFTeX\else
  % xetex/luatex font selection
\fi
% Use upquote if available, for straight quotes in verbatim environments
\IfFileExists{upquote.sty}{\usepackage{upquote}}{}
\IfFileExists{microtype.sty}{% use microtype if available
  \usepackage[]{microtype}
  \UseMicrotypeSet[protrusion]{basicmath} % disable protrusion for tt fonts
}{}
\makeatletter
\@ifundefined{KOMAClassName}{% if non-KOMA class
  \IfFileExists{parskip.sty}{%
    \usepackage{parskip}
  }{% else
    \setlength{\parindent}{0pt}
    \setlength{\parskip}{6pt plus 2pt minus 1pt}}
}{% if KOMA class
  \KOMAoptions{parskip=half}}
\makeatother
\usepackage{xcolor}
\usepackage[margin=1in]{geometry}
\usepackage{color}
\usepackage{fancyvrb}
\newcommand{\VerbBar}{|}
\newcommand{\VERB}{\Verb[commandchars=\\\{\}]}
\DefineVerbatimEnvironment{Highlighting}{Verbatim}{commandchars=\\\{\}}
% Add ',fontsize=\small' for more characters per line
\usepackage{framed}
\definecolor{shadecolor}{RGB}{248,248,248}
\newenvironment{Shaded}{\begin{snugshade}}{\end{snugshade}}
\newcommand{\AlertTok}[1]{\textcolor[rgb]{0.94,0.16,0.16}{#1}}
\newcommand{\AnnotationTok}[1]{\textcolor[rgb]{0.56,0.35,0.01}{\textbf{\textit{#1}}}}
\newcommand{\AttributeTok}[1]{\textcolor[rgb]{0.13,0.29,0.53}{#1}}
\newcommand{\BaseNTok}[1]{\textcolor[rgb]{0.00,0.00,0.81}{#1}}
\newcommand{\BuiltInTok}[1]{#1}
\newcommand{\CharTok}[1]{\textcolor[rgb]{0.31,0.60,0.02}{#1}}
\newcommand{\CommentTok}[1]{\textcolor[rgb]{0.56,0.35,0.01}{\textit{#1}}}
\newcommand{\CommentVarTok}[1]{\textcolor[rgb]{0.56,0.35,0.01}{\textbf{\textit{#1}}}}
\newcommand{\ConstantTok}[1]{\textcolor[rgb]{0.56,0.35,0.01}{#1}}
\newcommand{\ControlFlowTok}[1]{\textcolor[rgb]{0.13,0.29,0.53}{\textbf{#1}}}
\newcommand{\DataTypeTok}[1]{\textcolor[rgb]{0.13,0.29,0.53}{#1}}
\newcommand{\DecValTok}[1]{\textcolor[rgb]{0.00,0.00,0.81}{#1}}
\newcommand{\DocumentationTok}[1]{\textcolor[rgb]{0.56,0.35,0.01}{\textbf{\textit{#1}}}}
\newcommand{\ErrorTok}[1]{\textcolor[rgb]{0.64,0.00,0.00}{\textbf{#1}}}
\newcommand{\ExtensionTok}[1]{#1}
\newcommand{\FloatTok}[1]{\textcolor[rgb]{0.00,0.00,0.81}{#1}}
\newcommand{\FunctionTok}[1]{\textcolor[rgb]{0.13,0.29,0.53}{\textbf{#1}}}
\newcommand{\ImportTok}[1]{#1}
\newcommand{\InformationTok}[1]{\textcolor[rgb]{0.56,0.35,0.01}{\textbf{\textit{#1}}}}
\newcommand{\KeywordTok}[1]{\textcolor[rgb]{0.13,0.29,0.53}{\textbf{#1}}}
\newcommand{\NormalTok}[1]{#1}
\newcommand{\OperatorTok}[1]{\textcolor[rgb]{0.81,0.36,0.00}{\textbf{#1}}}
\newcommand{\OtherTok}[1]{\textcolor[rgb]{0.56,0.35,0.01}{#1}}
\newcommand{\PreprocessorTok}[1]{\textcolor[rgb]{0.56,0.35,0.01}{\textit{#1}}}
\newcommand{\RegionMarkerTok}[1]{#1}
\newcommand{\SpecialCharTok}[1]{\textcolor[rgb]{0.81,0.36,0.00}{\textbf{#1}}}
\newcommand{\SpecialStringTok}[1]{\textcolor[rgb]{0.31,0.60,0.02}{#1}}
\newcommand{\StringTok}[1]{\textcolor[rgb]{0.31,0.60,0.02}{#1}}
\newcommand{\VariableTok}[1]{\textcolor[rgb]{0.00,0.00,0.00}{#1}}
\newcommand{\VerbatimStringTok}[1]{\textcolor[rgb]{0.31,0.60,0.02}{#1}}
\newcommand{\WarningTok}[1]{\textcolor[rgb]{0.56,0.35,0.01}{\textbf{\textit{#1}}}}
\usepackage{graphicx}
\makeatletter
\def\maxwidth{\ifdim\Gin@nat@width>\linewidth\linewidth\else\Gin@nat@width\fi}
\def\maxheight{\ifdim\Gin@nat@height>\textheight\textheight\else\Gin@nat@height\fi}
\makeatother
% Scale images if necessary, so that they will not overflow the page
% margins by default, and it is still possible to overwrite the defaults
% using explicit options in \includegraphics[width, height, ...]{}
\setkeys{Gin}{width=\maxwidth,height=\maxheight,keepaspectratio}
% Set default figure placement to htbp
\makeatletter
\def\fps@figure{htbp}
\makeatother
\setlength{\emergencystretch}{3em} % prevent overfull lines
\providecommand{\tightlist}{%
  \setlength{\itemsep}{0pt}\setlength{\parskip}{0pt}}
\setcounter{secnumdepth}{-\maxdimen} % remove section numbering
\usepackage{fvextra} \DefineVerbatimEnvironment{Highlighting}{Verbatim}{breaklines,commandchars=\\\{\}}
\ifLuaTeX
  \usepackage{selnolig}  % disable illegal ligatures
\fi
\IfFileExists{bookmark.sty}{\usepackage{bookmark}}{\usepackage{hyperref}}
\IfFileExists{xurl.sty}{\usepackage{xurl}}{} % add URL line breaks if available
\urlstyle{same}
\hypersetup{
  pdftitle={Lab 11 Exercises},
  hidelinks,
  pdfcreator={LaTeX via pandoc}}

\title{Lab 11 Exercises}
\author{}
\date{\vspace{-2.5em}2022-11-13}

\begin{document}
\maketitle

\hypertarget{instructions}{%
\subsection{Instructions}\label{instructions}}

\begin{itemize}
\tightlist
\item
  I will post a key a week from now
\item
  If you'd like to get extra credit you must submit your answers AND
  code in this format on Moodle.
\item
  Submit your pdf and point me to where I can find your .Rmd and compile
  it
\item
  You must do this before I post the key
\end{itemize}

\hypertarget{contingency-analysis-problems}{%
\subsubsection{Contingency analysis
problems}\label{contingency-analysis-problems}}

\begin{enumerate}
\def\labelenumi{\arabic{enumi}.}
\tightlist
\item
  Spousal Grief (Chapter 9, problem 29)
\end{enumerate}

It is common wisdom that death of a spouse can lead to health
deterioration of the partner left behind. Is common wisdom right or
wrong in this case? To investigate, Maddison and Viola (1968) measured
the degree of health deterioration of 132 widows in the Boston area, all
of whose husbands had died at the age of 45--60 within a fixed six-month
period before the study. A total of 28 of the 132 widows had experienced
a marked deterioration in health, 47 had seen a moderate deterioration,
and 57 had seen no deterioration in health. Of 98 control women with
similar characteristics who had not lost their husbands, 7 saw a marked
deterioration in health over the same time period, 31 experienced a
moderate deterioration of health, and 60 saw no deterioration.

\begin{enumerate}
\def\labelenumi{\Alph{enumi})}
\item
  Test whether the pattern of health deterioration was different between
  the two groups of women. Give the as precisely as possible from the
  statistical tables.
\item
  Now use R to check your answer:
\end{enumerate}

\begin{Shaded}
\begin{Highlighting}[]
\CommentTok{\#your code}
\end{Highlighting}
\end{Shaded}

\begin{enumerate}
\def\labelenumi{\Alph{enumi})}
\setcounter{enumi}{2}
\tightlist
\item
  Interpret your result in words:
\end{enumerate}

\begin{enumerate}
\def\labelenumi{\arabic{enumi}.}
\setcounter{enumi}{1}
\tightlist
\item
  Kuru disease (Chapter 9, problem 35)
\end{enumerate}

Kuru is a prion disease of the Fore people of highland New Guinea. It
was once transmitted by the consumption of deceased relatives at
mortuary feasts, a ritual that was ended by about 1960. Using archived
tissue samples, Mead et al.~(2009) investigated genetic variants that
might confer resistance to kuru. The data in the accompanying table are
genotypes at codon 129 of the prion protein gene of young and elderly
individuals all having the disease. Since the elderly individuals have
survived long exposure to kuru, unusually common genotypes in this group
might indicate resistant genotypes. The file is
\texttt{input\_files/kuru.csv}.

\begin{enumerate}
\def\labelenumi{\Alph{enumi})}
\tightlist
\item
  Illustrate these data with a grouped bar graph. Which genotype(s) are
  especially prevalent in the elderly compared with young individuals?
\end{enumerate}

\begin{Shaded}
\begin{Highlighting}[]
\CommentTok{\#your code}
\end{Highlighting}
\end{Shaded}

Answer:

\begin{enumerate}
\def\labelenumi{\Alph{enumi})}
\setcounter{enumi}{1}
\tightlist
\item
  Test whether genotype frequencies differ between the two age groups.
\end{enumerate}

\begin{Shaded}
\begin{Highlighting}[]
\CommentTok{\#your code}
\end{Highlighting}
\end{Shaded}

Answer:

\begin{center}\rule{0.5\linewidth}{0.5pt}\end{center}

\hypertarget{using-webapps-to-get-a-feel-for-the-normal-distribution-and-the-central-limit-theorem}{%
\subsubsection{Using webapps to get a feel for the normal distribution
and the central limit
theorem}\label{using-webapps-to-get-a-feel-for-the-normal-distribution-and-the-central-limit-theorem}}

We return to the applet we used in tutorial 3, located at
\url{http://www.zoology.ubc.ca/~whitlock/Kingfisher/SamplingNormal.htm}
to investigate three points made in the text.

\begin{itemize}
\item
  Point 1: The distribution of sample means is normal, if the variable
  itself has a normal distribution. First, hit ``COMPLETE SAMPLE OF 10''
  and ``CALCULATE MEAN'' a few times, to remind yourself of what this
  applet does. (It takes a sample from the normal distribution shown
  when you click ``SHOW POPULATION''. The top panel shows a histogram of
  that sample, and the bottom panel shows the distribution of sample
  means from all the previous samples.)
\item
  Next, hit the ``MEANS FOR MANY SAMPLES'' button. This button makes a
  large number of separate samples at one go, all of the same sample
  size, to save you from making the samples one by one. Notice that the
  sample mean differs from sample to sample. The sample mean produced by
  random sampling from a probability distribution is itself a random
  variable.
\end{itemize}

\begin{enumerate}
\def\labelenumi{\Alph{enumi})}
\tightlist
\item
  Look at the distribution of sample means. Does it seem to have a
  normal distribution? Click the checkbox by ``SHOW SAMPLING
  DISTRIBUTION'' off to the right, which will draw the curve for a
  normal distribution with the same mean and variance as this
  distribution of sample means.
\end{enumerate}

Answer:

\begin{itemize}
\tightlist
\item
  Point 2: The standard deviation of the distribution of sample means is
  reduced with larger sample sizes. The standard deviation of the
  population is controlled by the right slider marked with \(\sigma\).
  The sample size is set by left slider. (The default when it opens is
  set to n=10.)
\end{itemize}

\begin{enumerate}
\def\labelenumi{\Alph{enumi})}
\setcounter{enumi}{1}
\tightlist
\item
  For \(n=10\), have the applet calculate a large number of sample means
  as you did in the previous exercise. If each sample size is 10 and the
  standard deviation is 30 (as in the default). What do you predict the
  standard deviation of the sample means to be? (Use the equation you
  have learned in class to make this calculation.)
\end{enumerate}

Answer:

\begin{enumerate}
\def\labelenumi{\Alph{enumi})}
\setcounter{enumi}{2}
\tightlist
\item
  Change the sample size to \(n=100\), and recalculate many sample
  means. Calculate the predicted standard deviation of all the sample
  means. Should the sampling distribution of sample means be wider or
  narrower with this larger sample size than in the previous case with a
  smaller sample size? What do you observe in the simulations?
\end{enumerate}

\begin{itemize}
\tightlist
\item
  Point 3: The distribution of sample means is approximately normal no
  matter what the distribution of the variable, as long as the sample
  size is large enough. (The Central Limit Theorem) Load another web
  page: \url{http://www.zoology.ubc.ca/~whitlock/Kingfisher/CLT.htm}
\end{itemize}

This will simulate a very skewed distribution of data, showing the
number of cups of coffee drunk per week for a population of university
students. Click on ``COFFEE'' to see the distribution of the variable
among individual students. Describe the ways that this looks different
from a normal distribution.

\begin{enumerate}
\def\labelenumi{\Alph{enumi})}
\setcounter{enumi}{3}
\tightlist
\item
  Set \(n=2\) for the sample size, and simulate many sample means. Does
  the distribution of sample means look normal? Is it closer to normal
  in its shape than the distribution of individuals in the population?
\end{enumerate}

Answer:

\begin{enumerate}
\def\labelenumi{\Alph{enumi})}
\setcounter{enumi}{4}
\tightlist
\item
  Now set \(n=25\) and simulate many sample means. How does the
  distribution of sample means look now? It should look much more like a
  normal distribution, because of the Central Limit Theorem.
\end{enumerate}

\begin{center}\rule{0.5\linewidth}{0.5pt}\end{center}

\hypertarget{normal-distribution-exercises}{%
\subsubsection{Normal Distribution
exercises}\label{normal-distribution-exercises}}

1.Let's use R's random number generator for the normal distribution to
build intuition for how to view and interpret histograms and QQ plots.
Remember, the lists of values generated by rnorm() come from a
population that truly have a normal distribution.

\begin{enumerate}
\def\labelenumi{\Alph{enumi})}
\tightlist
\item
  Generate a list of 10 random numbers from a normal distribution with
  mean 15 and standard deviation 3 and save the results to
  \texttt{normal\_vector}:
\end{enumerate}

\begin{Shaded}
\begin{Highlighting}[]
\CommentTok{\#your code}
\end{Highlighting}
\end{Shaded}

\begin{enumerate}
\def\labelenumi{\Alph{enumi})}
\setcounter{enumi}{1}
\tightlist
\item
  Use hist() to plot a histogram of these numbers from part a.
\end{enumerate}

\begin{Shaded}
\begin{Highlighting}[]
\CommentTok{\#your code}
\end{Highlighting}
\end{Shaded}

\begin{enumerate}
\def\labelenumi{\Alph{enumi})}
\setcounter{enumi}{2}
\tightlist
\item
  Plot a QQ plot from the numbers in part a.
\end{enumerate}

\begin{Shaded}
\begin{Highlighting}[]
\CommentTok{\#your code}
\end{Highlighting}
\end{Shaded}

\begin{enumerate}
\def\labelenumi{\Alph{enumi})}
\setcounter{enumi}{3}
\tightlist
\item
  Repeat a through c several times (at least a dozen times). For each,
  look at the histograms and QQ plots. Think about the ways in which
  these look different from the expectation of a normal distribution
  (but remember that each of these samples comes from a truly normal
  population).
\end{enumerate}

\begin{Shaded}
\begin{Highlighting}[]
\CommentTok{\#your code}
\end{Highlighting}
\end{Shaded}

Observations:

\begin{enumerate}
\def\labelenumi{\arabic{enumi}.}
\setcounter{enumi}{1}
\tightlist
\item
  Repeat the procedures of Question 1, except this time have R sample
  250 individuals for each sample. (You can use the same command as in
  Question 1, but now set n = 250.) Do the graphs and QQ plots from
  these larger samples look more like the normal expectations than the
  smaller sample you already did? Why do you think that this is?
\end{enumerate}

\begin{Shaded}
\begin{Highlighting}[]
\CommentTok{\#your code}
\end{Highlighting}
\end{Shaded}

Observations:

\begin{enumerate}
\def\labelenumi{\arabic{enumi}.}
\setcounter{enumi}{2}
\tightlist
\item
  In 1898, Hermon Bumpus collected house sparrows that had been caught
  in a severe winter storm in Chicago. He made several measurements on
  these sparrows, and his data are in the file ``bumpus.csv''.
\end{enumerate}

Bumpus used these data to observe differences between the birds that
survived and those that died from the storm. This became one of the
first direct and quantitative observations of natural selection on
morphological traits. Here, let's use these data to practice looking for
fit of the normal distribution. (We'll return to this data set next week
to look for evidence of natural selection.)

\begin{enumerate}
\def\labelenumi{\Alph{enumi})}
\tightlist
\item
  Use ggplot() to plot the distribution of total length (this is the
  length of the bird from beak to tail). Does the data look as though it
  comes from distribution that is approximately normal?
\end{enumerate}

Answer: B) Use qqnorm() to plot a QQ plot for total length. Does the
data fall approximately along a straight line in the QQ plot? If so,
what does this imply about the fit of these data to a normal
distribution?

Answer: C) Calculate the mean of total length and a 95\% confidence
interval for this mean. (You may want to refer back to Week 5 for the R
commands to do this.)

\begin{Shaded}
\begin{Highlighting}[]
\CommentTok{\#your code}
\end{Highlighting}
\end{Shaded}

Answer:

For the problems below, use both:

\begin{enumerate}
\def\labelenumi{\arabic{enumi}.}
\item
  the ``manual'' route (using R only as a calculator) and
\item
  using R built-in normal distribution functions (use this to check your
  first answer)
\item
  The gestation period for cats has an approximate mean of 64 days and a
  standard deviation of 3 days, and the distribution of the gestation
  period is approximately Normal. What gestation period best corresponds
  to the 25th percentile?
\end{enumerate}

\begin{Shaded}
\begin{Highlighting}[]
\CommentTok{\#your code}
\end{Highlighting}
\end{Shaded}

Answer:

\begin{enumerate}
\def\labelenumi{\arabic{enumi}.}
\setcounter{enumi}{4}
\tightlist
\item
  The gestation period for cats has an approximate mean of 64 days and a
  standard deviation of 3 days, and the distribution of the gestation
  period is approximately Normal. What gestation period corresponds to
  the top 10\% of gestation periods? Round to the nearest tenth of a
  day.
\end{enumerate}

\begin{Shaded}
\begin{Highlighting}[]
\CommentTok{\#your code}
\end{Highlighting}
\end{Shaded}

Answer:

\begin{enumerate}
\def\labelenumi{\arabic{enumi}.}
\setcounter{enumi}{5}
\tightlist
\item
  The gestation period for cats has an approximate mean of 64 days and a
  standard deviation of 3 days, and the distribution of the gestation
  period is approximately Normal. What proportion of kittens have a
  gestation period between 62 days and 70 days? Round to two decimal
  places.
\end{enumerate}

\begin{Shaded}
\begin{Highlighting}[]
\CommentTok{\#your code}
\end{Highlighting}
\end{Shaded}

Answer:

\begin{enumerate}
\def\labelenumi{\arabic{enumi}.}
\setcounter{enumi}{6}
\tightlist
\item
  The gestation period for cats has an approximate mean of 64 days and a
  standard deviation of 3 days, and the distribution of the gestation
  period is approximately Normal. What proportion of kittens have a
  gestation period longer than 62 days? Round your answer to two decimal
  places.
\end{enumerate}

\begin{Shaded}
\begin{Highlighting}[]
\CommentTok{\#your code}
\end{Highlighting}
\end{Shaded}

Answer:

\begin{enumerate}
\def\labelenumi{\arabic{enumi}.}
\setcounter{enumi}{7}
\tightlist
\item
  The length of time before a seed germinates when falling on fertile
  soil is approximately Normally distributed with a mean of 600 hours
  and a standard deviation of 100 hours. What is the probability that a
  seed will take more than 720 hours before germinating? Round to two
  decimal places.
\end{enumerate}

\begin{Shaded}
\begin{Highlighting}[]
\CommentTok{\#your code}
\end{Highlighting}
\end{Shaded}

Answer:

\begin{enumerate}
\def\labelenumi{\arabic{enumi}.}
\setcounter{enumi}{8}
\tightlist
\item
  Using fluorescent imaging techniques, researchers observed that the
  position of binding sites on HIV peptides is approximately Normally
  distributed with a mean of 2.45 microns and a standard deviation of
  0.35 micron. What is the standardized score for a binding site
  position of 2.03 microns? (Enter your answer rounded to one decimal
  place.)
\end{enumerate}

\begin{Shaded}
\begin{Highlighting}[]
\CommentTok{\#your code}
\end{Highlighting}
\end{Shaded}

Answer:

10.Scientists discovered a new group of proteins in an animal species.
They found that the distribution of the number of amino acids these
proteins were made of was approximately Normal with mean 530 and
standard deviation 80. About what percent of these new proteins will be
between 490 and 590 amino acids long?

\begin{Shaded}
\begin{Highlighting}[]
\CommentTok{\#your code}
\end{Highlighting}
\end{Shaded}

Answer:

\begin{enumerate}
\def\labelenumi{\arabic{enumi}.}
\setcounter{enumi}{10}
\tightlist
\item
  An 1868 paper by German physician Carl Wunderlich reported, based on
  over a million body temperature readings, that healthy‑adult body
  temperatures are approximately Normal with mean 𝜇=98.6 degrees
  Fahrenheit (∘F) and standard deviation 𝜎=0.6∘F . This is still the
  most widely quoted result for human temperature.
\end{enumerate}

\begin{enumerate}
\def\labelenumi{\Alph{enumi})}
\tightlist
\item
  According to this study, what is the range of body temperatures that
  can be found in 95\% of healthy adults? (We are looking for the middle
  95\% of the adult population.)
\end{enumerate}

\begin{Shaded}
\begin{Highlighting}[]
\CommentTok{\#your code}
\end{Highlighting}
\end{Shaded}

Answer: B) According to this study, what is the variance in normal human
temperature in Celsius. Use the function you wrote previously.

\begin{Shaded}
\begin{Highlighting}[]
\CommentTok{\#your code}
\end{Highlighting}
\end{Shaded}

Answer:

\end{document}
